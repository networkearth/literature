\documentclass[11pt,a5paper]{book}
\usepackage[utf8]{inputenc}
\usepackage{amsmath}
\usepackage{amsfonts}
\usepackage{amssymb}
\usepackage{graphicx}
\usepackage[super]{nth}

\title{Way Points}
\author{Marcel Gietzmann-Sanders}
\begin{document}
\maketitle
\tableofcontents
\newpage
\chapter{Network Earth}
\section{The Biosphere}
Take a breath. 
\newline

Washing into your lungs is a rich mixture of gasses containing the all important and life giving element - oxygen. While you may traditionally think of oxygen as coming from the plants around you, it turns out that the mixture swirling about in your lungs comes from a much wider and more exotic set of sources. The majority of the oxygen you are breathing comes from our oceans - one in five breaths from a cyanobacterium named Prochlorococcus \cite{kmorsink}. On the complete other end of spectrum are rain-forests like the Amazon - a place that produces something close to one in ten of your breaths \cite{ymalhi}. The result of this is that the brew of gasses swirling around in your lungs has come from creatures smaller than the single strands of hair on your head to those as tall as multistory buildings. However, the story doesn't end there.
\newline

If you're near some kind of urban center, you probably deal with some level of pollution. The daily goings of urban living produce loads of pollutants such as nitrogen and sulfur dioxide. However, plants have a habit of pulling such pollutants out of the air and storing them safely away in their tissues. In fact a 2006 estimate found that in the United States alone urban forests were responsible for more than 3 billion dollars of air purification \cite{dnowak}. So while most of your oxygen was sourced from our oceans and rainforests, it's purification was local and done by the trees and shrubs in your back yard. 
\newline

Take a bite. 
\newline 

If you happen to be eating cassava from South East Asia then you can thank a tiny parasitic wasp - \textit{Anagyrus lopezi} - that is reared in the millions to control populations of a devastating pest - the mealybug \cite{wpark}. An enormous boon for both the financial and food security in the region, the economic value of this millimeter long wasp is more than \$14 billion a year. 
\newline

If instead you're dining on a lovely, flaky slab of wild salmon, you are effectively eating the entire ocean. As predators that sit relatively high in the food chain, they depend on "bait fish" like herring as their primary source of food. Herring feed on things like krill, who in turn eat the very plankton providing us with the majority of our oxygen. If any one of these levels in the food chain starts to experience issues so too does your flaky slab of salmon. 
\newline

Finally, if there are plants on your plate there's a reasonable chance you can thank the leagues of beetles, flies, butterflies, and wasps managing pollination in our world. These services are supplied to an estimated 75\% of crop species, and are worth something upwards of \$200 billion every year \cite{avanbergen}! Without the insects pollinating the plants that produce our food the variety in our diet would shrink dramatically.
\newline

Go on vacation. 
\newline

Diving at a coral reef? The colors, the vivacity, the sheer brilliance of all the fish darting this way and that around you? It's thanks to the sharks. Sharks specifically target sick or unhealthy fish, thereby improving the overall health of fish populations and possibly preventing outbreaks of disease \cite{reefcause}. Furthermore, they help prevent the kinds of population explosions that lead to mass die offs. Just another way they preserve the fish we all love seeing. 
\newline

Hiking alongside giant sequoias or passing through the incredibly diverse south african fynbos? Both of these are only possible thanks to the drama of fire and lightning. Sequoias depend on fires to open their tightly sealed cones and the resultant sprouts depend on the brush clearing power of the fires to give them any chance of seeing the sun \cite{california}. Then, of the more than 9,000 plant species found in South Africa alone, many, like the proteas, also depend on fire to release their seeds and prepare the ground for the next generation of amazing \cite{shoek}. So why do these fires occur? Lightning is the spark, but the reason the fires spread is thanks to the vast diversity of plants that leave behind dried out remains. When it comes to sequoias and proteas, it truly takes a village. 
\newline

Enjoying all the ocean has to offer in Florida? Thank the mangroves. Not only do mangroves provide nurseries for an incredible diversity of creatures including the fish that crowd coral reefs and the manatees and birds that people flock to see (pun intended), but they are also veritable buttresses against coastal damage. One estimate suggested that during Huricane Irma in 2017 they prevented over a billion dollars of flood damage! That's a lot of insurance. 
\newline

Take a step back. 
\newline

These examples are just a tiny taste of how \textit{every single aspect of our lives} is deeply, intrinsically dependent on the living world around us - the biosphere. The Millenium Ecosystem Assessment identified 25 different categories of ecosystem services - and these are as broad as things like food provisioning and erosion control \cite{mas}! From millimeter wasps to three hundred foot tall trees, from ecosystems half a world away to the shrubs in your back yard, our world provides our food, water, air, health, happiness, and so much more. It is our life support system, our spaceship as we hurtle through the vast, cold emptiness that is space and for the first time in our history we have the power to overwhelm it. 

\section{The Anthropocene}
Let's start with some perspective. We as human beings have a very hard time thinking about pretty much any period of time longer than the ones we've already lived. Tell me something is ten years old - no problem I can grasp that. Tell me something is one thousand years old? While I can make it out mathematically I really don't have any way to grasp just how much longer that is. 100,000 years? Forget about it. So let's use a frame of reference that's a little easier to understand. Let's pretend all of our universe's history fits in a single year \cite{csagan}. 

From this point of view (and if we accept standard scientific time lines) the big bang starts us off at midnight January 1. The next four months are spent in a dizzying dance of expanding hot gas, newly forming stars, and, finally, galaxy formation with the milky way only showing up around May \cite{eellis}. But don't assume that because of this our time line is about to speed up because we still have to wait until September before our solar system even forms! We've gotten roughly three quarters through our year and our planet has only \textit{just} shown up. Surprisingly, given how long it's taken to get to this point, it only takes 20 days for the first things we'd deem living to show up, but then things slow down again as we have to wait until December (around 70 more days) before the first multi-cellular organism joins the scene. Now things really start to take off. 

15 days later we've got plants colonizing the land masses of our planet. 3 days after that the first reptiles join them. Only a mere two days later and we've entered the age of the dinosaurs. Think about that, in less than a third of the time it took to figure out how to make multi-cellular organisms we've gone from nothing macroscopic on land to dinosaurs! A day later, our ancestors - the first mammals - join the production and by December 30th (4 days later) the first primates appear. 

But hold on one second. Did I just say December 30th? Does that mean we've got 1 day left in our year and humans haven't even shown up?! That's right. But it gets even wilder. Homo sapiens, us, don't show up until 12 \textit{minutes} to midnight. Just chew on that for a second. All of human history, finding fire, inventing tools, starting the first cities, all the empires, all the wars, absolutely \textit{everything} that our species has ever done occurred in the last 12 minutes of our universe's year. That's the amount of time it takes to heat up a microwave dinner!

But it gets even wilder because we, as a species, didn't figure out how to farm until 28 \textit{seconds} before midnight. So really, everything you've learned about in your human history courses gets packed into about as much time as you spend watching a \textit{single} tiktok. We are but a blip in the history of our universe. But, oh, what a blip we are. 
\newline

From nature's perspective, humans have generally been, with few exceptions, a footnote. There's a reason why the oceans were terrifying to early humans, or why people prayed to gods to stave off floods or famine. If nature decided to have its way, there was little we could do about it. But post the industrial revolution and especially since 1950, that relationship has \textit{utterly} changed \cite{eellis}.

Let's start with some simple numbers. Around 71\% of the land on our planet is considered habitable \cite{hritchie}. Of that habitable land we have put around 46\% of it into agricultural production. That means we use a solid \textit{third} of the planet's surface (not covered by ocean) to just feed ourselves. Except that we eat more than what's on the land. When it comes to fishing the picture is even wilder. Of global fish stocks we overfish 34\% percent and maximally fish 60\% \cite{mroser}. That means there are only 6\% of fish stocks that we could fish more without starting to deplete their numbers!

The amount of water we use is no less incredible than the amount of area we farm. Each year approximately 40,000 $km^3$ of freshwater flows into the ocean \cite{eellis}. Of this only around 13,000 $km^3$ is accessible to humans with our current technology. Yet nearly half of that flows through human engineered systems! So not only are we using half of the accessible land, and pretty much all of the accessible fish, we're using half the available fresh water too. 

In the 2004 the International Geosphere-Biosphere Programme published a report that illustrated what is now called the Great Acceleration \cite{wsteffen}. Across 24 distinct measures from total GDP to atmospheric carbon to the number of rivers dammed they found an explosive spike in activity starting in the 1950's. We've already talked about a few of them (farm, fish, and water use) but let's sprinkle a few more in for color. 

Since around the beginning of the 1900's we've lost 30\% of our forests. We've now dammed close to 25 thousand rivers. We consume about 300 million tons of fertilizer per year. Our global population has gone from 2 to 7 billion. And we have more than 3 times as many floods as we used to per decade. But here's the one that kind of summarizes it all for me - extinction rates have gone through the roof.  
\newline

Extinctions are nothing new and in general pretty normal. As time marches on species come and go and that's... well... just evolution doing its thing. So what matters is not \textit{whether} extinctions are happening but \textit{how many} are happening. 

In the past we've had some pretty spectacular extinction events. Around 444 million years ago the Ordovician-Silurian extinction took out around 70\% of everything living at the time \cite{adubey}. 372 million years ago the Late Devonian extinction took out some 70\% of all marine species. The Permian-Triassic extinction, which occurred roughly 252 million years ago, took out over 70\% of everything alive again. Then in the Late Triassic (208 million years ago) we had yet another mass extinction. Finally we have the most famous of them all, the Cretaceous-Paleogene extinction where 66 million years ago over 60\% of all species died as the result of a massive asteroid smashing into our planet and creating the equivalent of nuclear fallout. Bad for the dinosaurs but great for us as mammals often are attributed their place center stage thanks to that big old space rock. 

Since we were visited by Armageddon-a-la-space extinction rates have pretty much held at their normal, background levels. Until, that is, the Great Acceleration. Since the 1950's extinction rates have skyrocketed to more than 1000 times their background rate \cite{wwf}! At the current rate scientists estimate that 50\% of all species could be gone by 2100. We have entered the sixth extinction. 

Here is where our current impact on our biosphere really comes into perspective. Mass extinctions in the past have occurred as the result of insanely powerful events. Country sized asteroids smashing into the planet. Massive anomalies in global temperatures (sound familiar?). Such incredible dips in the amount of oxygen that more than 70\% of all living species just... well... died of oxygen deprivation. Causing mass extinctions requires planet shaping power. \textit{We now possess that kind of power}. And as a result we've entered an entirely new geological epoch, an epoch where humankind is the domineering, world shaping, power - the Anthropocene \cite{eellis}. And in the perspective of our "universe in a year" we managed this power grab in less time than it takes to watch a tiktok.
\newline

Contrary to how you may be feeling, none of this is meant to scare you (although fear and awe are perfectly appropriate reactions). Instead it's meant to impress upon you the gravity of our situation. In the first section we got a glimpse of just how important the biosphere is to us and now you've got a taste of just how powerful we, as a species, have become - that asteroid that took out the dinosaurs is taking notes from us now. Thankfully this power means that in so much as we can absolutely wreck our world, we can also take care of it too. The problem is not that our world is doomed, but rather that we can no longer just take it for granted. Given our planet shaping capabilities we have no choice but to become active, conscious stewards of our planet. So the question is simple, how do we do that?

\section{A Revolution}
Suppose for a moment you have a garden. While at first it may seem like all you need to do is water your plants and let them flourish, anyone who's tried gardening before knows there's a lot more to it than that. For one thing watering is a tricky business. Water too much and your plants can get water logged and die. Water too little and your plants will wither up and die. Plant your plants too close together and they won't be able to grow properly. Plant them too far apart and you'll be wasting loads of garden space. Then there are the pests. Caterpillars, beetles, mice, deer, you name it it's probably trying to eat your plants and every single one of these needs a different strategy for dealing with. Then there's disease. Plants, like us, can get sick and it's especially likely if they're not getting the right balance of nutrients and care. Diagnosing these sicknesses is difficult enough that there are apps to help you with the task and then there are a whole slew of different methodologies to deal with the various ailments. And we haven't even gotten to the fact that choosing plants can be its own nightmare. Do you have enough sun? Too much sun? Direct or indirect? What's the soil quality, what's your growing zone? On and on it goes. Point is, stewardship of your garden requires a lot of specific knowledge about the plants you a growing, their life histories, the life histories of everything they interact with, the place you live in, and so on. 

So now replace your garden with the entire world. Obviously the amount of knowledge you need is going to skyrocket but something else changes as well. In your garden you've got lots of room to learn by trial and error - if you manage to kill a plant (like I tend to do) it's no big deal, just try again next year or in a different plot in your garden. We, however, have only one world. While we can certainly try little things here and there, at the end of the day, we've got to be pretty sure we know what we're doing before we launch large scale efforts to shape our world one way or another. And that requires models sophisticated enough to let us trial things in simulation. So not only do we need a lot of knowledge about how our biosphere works, we also need to make sure it's the kind of knowledge that allow us to build simulations. Turns out, this kind of science is pretty new. 
\newline

Some history. The scientific revolution started in the 1500's and was set off by Copernicus' now famous idea that perhaps we aren't at the center of the universe \cite{sbrush}. What followed was a massive explosion in all things astronomy and over the course of a few hundred years everything we thought we knew about our place in the universe changed. In 1620 Robert Hooke discovered cells for the first time launching the whole field of microbiology \cite{wsd}. In 1735 Carl Linnaeus put together the system for classifying plants and animals that we use to this day and in 1745 Ewald Jürgen Georg von Kleist invented the first capacitor. Fast forwarding to the 1900's and we've got Max Planck's theory of black body radiation in 1900 followed quickly by Einstein's special relatively and his explanation of the photoelectric effect in 1915 - all theories that were seminal in the creation of modern physics. In 1953 Wilkins, Franklin, Watson, and Crick made the now famous discovery that DNA has a helical structure - thereby totally changing our understanding of the stuff - and in 1996 we cloned a sheep for the first time. So, given all of this progress, you might be surprised to learn that it was only in 1999 that Hans Schellnhuber published a "groundbreaking" paper on earth system modeling that spoke of a "second Copernican revolution" in our ability to model the biosphere \cite{hschellnhuber}. Our first thoughts on this issue came around the same time as our ability to clone things and well after we came up with the idea of black holes in space. 

To understand this we need to look at what was going on alongside each of these revolutions in science. The scientific revolution, with all of its progress in astronomy, came at the same time as the first telescopes were being invented. Hooke made his discoveries about the cell thanks to the newly invented compound microscope. Linnaeus only came up with his system of taxonomy after extensively traveling the world - something only possible thanks to all the nautical inventions that also created the "age of discovery". The first electrical inventions were all thanks to developments in what materials were becoming available as the industrial revolution unfolded \cite{tkuhn}. The discoveries of modern physics were all thanks to continuous improvements in our ability to poke at smaller and smaller things using tools like the x-ray diffraction techniques that allowed Wilkins, Franklin, Watson, and Crick to infer the structure of DNA. The pattern from all of these examples should be pretty clear - to understand our world we need the tools to probe it and each scientific revolution has been attended by updated capabilities in our ability to gather data \cite{tkuhn}. So it should come as no surprise then that one of the major tools used by earth system modelers - satellites - are a relatively recent development. As an example, the Landsat program, one of the largest programs to provide satellite imagery of Earth, only started in 1972 \cite{wls}.

But satellites are only part of the picture and the need for further instrumentation and tooling is clear. Whether it's land management \cite{jpongratz}, pollination services \cite{ibartomeus}, or general biodiversity management \cite{hkuhl} (just to name a few) scientists in recent years continue to point to a lack of comprehensive data as one of the primary challenges they face world over. Science is driven by data, and the data is sorely lacking. 
\newline

So if we want to spark another scientific revolution, one that will give us the tools we need to be good stewards of our planet, we need to start providing the instrumentation required. So question is - how do we do it?

\section{An Aside}
Before we dive into how to develop instrumentation I want to stress something. The last section argues that we \textit{need} models of our biosphere and all the attendant knowledge in order to be good stewards of our planet. I firmly believe this is true. However (and this is important) I \textit{do not} believe that such knowledge should block actions that are already supported by the science that exists \textit{today}. I \textit{do not} believe we should wait to decarbonize or wait to stop wasting as much as we do or wait to stop polluting. These are all good, clear actions that should be taken \textit{now} because the science clearly backs them. In general there will always be gaps in our knowledge that need filling, so if we ever want to be good stewards who actually take action we have to learn to listen to the knowledge we have. 

Alright, back to the scheduled programming. 

\section{An Industry}
We've got a problem. While pointing out the immediate need for biosphere instrumentation is cool and all, it's also laughably vague. How on earth are we supposed to execute on a mission statement that's that high level? Fact is, we need a strategy. 

Strategies start with understanding our goals and, especially in our case, clarifying goals is really important. The history of conservation and "biosphere activism" has... well... a checkered past. Early biologists exploring the pacific islands sometimes caused extinctions in "the name of science" and part of the reason we're having to reintroduce wolves all over the place is that early conservationists helped get rid of them in the name of creating a more bounteous United States \cite{mnijhuis}. At the other extreme, in my own experience, I've found that frustration with human negligence often leads to a desire to take humans out of the picture entirely, a philosophy that culminated in the Half-Earth Hypothesis - a proposal that we should set aside half of the planet as a human free zone (and they're not talking about totally uninhabitable places) \cite{ewilson}. 

But let's look at intentions. The folks causing extinctions in the pacific islands? They just wanted to study birds. Unfortunately, at the time, studying birds was synonymous with shooting them. So while studying birds is a great idea, shooting them all as a means to an end is a little more than short sighted. 

How about the wolves? Well quite a lot of the money in conservation coffers comes from hunters. So making our forests more bounteous isn't a half bad idea. The problems are twofold. First, it turns out wolves are needed to keep deer populations healthy (ecology!). Second, extirpation is a really extreme take on "reducing" wolf populations. So once again, reasonable intentions with a short sighted strategy.

Finally I get the frustration with people seemingly just bulldozing everything that isn't "producing human good" or treating the natural world with less than respect. But the problem here is one of education - rather than understanding that rainforests are useful natural resources, most economies see them as unrealized cattle farms instead. The problem isn't that humans are included, it's that those same humans aren't including the natural world in their calculus. 

Regardless of the various strategies taken, one thing is clear from all of these examples. The goal is to nurture the value that our biosphere provides us. Unfortunately this is still too vague for our purposes so let's dig deeper.
\newline

The Millennium Ecosystem Assessment \cite{mes} suggests a rather lovely framework for thinking about all of this. They suggest (and the suggestion is backed by data) that human well being exists on a continuous spectrum with poverty that is made up of five key components - the necessary material for a good life, health, good social relations, security, and freedom and choice. So while you may have a lot of material wealth (1) if your health (2) is extremely poor because of, say, pollution, you are still impoverished to some degree. In other words there is no such thing as poor or wealthy. Instead there is a five dimensional spectrum that we have to keep our eyes on at all times. 

As a quick example of why this is important, almost every assessment of ecosystems that I have read is stated in dollars. Even the first section of this essay had a dollar bent! But how do you put dollar values on social relations, feelings of security, or autonomy? You obviously can't! Having these five dimensions of human well-being helps us break out of the habit of only seeing value as economic. And this level of specificity and holism is exactly what we need in our goals.
\newline

Okay so that gives us a better sense of how to make sure we're measuring value completely, but how on earth are we supposed to connect this back to the literal jungle's worth of complexity in our biosphere? Trying to work out all the ways in which our biosphere provides for, say, our health brings the phrase - boiling the ocean - to mind. Clearly we need to divide the problem up into more manageable parts. Once again, the Millennium Ecosystem Assessment is here to help \cite{mas}. 

As I mentioned before they have already done the work of divvying up ecosystem services into nice neat bins. At the top level we have three overarching categories - provisioning services, regulating services, and cultural services - which are each then divided up into specific sub categories such as food and fiber, water regulation, pollination, sense of place, disease regulation, and so forth. These, obviously, are much more manageable pieces of the overall biosphere pie.
\newline

Alright, let's bring this all back. How does this help clarify our path to building biospheric instrumentation? Well first we know now that there are 25 specific models that we are trying to build - one for each ecosystem service category. Second we know that in order to be complete, those models need to be capable of helping us predict the effects on each of the 5 dimensions of human well-being. Finally we know that those models need to be comprehensive. So we can clarify our path forward as such - we need to help scientists obtain the data required to build comprehensive models that connect each of the 25 ecosystem service categories outlined by the Millennium Ecosystem Assessment to the 5 dimensions of human well-being. Which leaves a much simpler question - what data is required?
\newline

Here, we come to the kicker. Scientists have not exactly been twiddling their thumbs. Huge strides have been taken in learning how to think about and model this kind of stuff. Remember, the issue scientists have is not in having \textit{no} data, it's in not having \textit{enough} of it. Pull down the journal Ecological Informatics and you will find \textit{hundreds} of examples of instrumentation techniques that scientists have already developed and proven useful. They've already done the de-risking work and know what data they need - they just need help deploying the instrumentation at scale! And doing this, would be a real value add. 

Think about it this way. Most of us could, if we really wanted to, wire our own homes, do our own plumbing, and so forth. But doing so requires buying all sorts of tools, learning all sorts of techniques, going through all sorts of safety training, and so on. That's a lot of overhead for one house. A professional electrician also has to go through all of this overhead but then they apply their tools and training to loads of homes. This is the economy of scale - and it makes a big difference. We absolutely need this same economy of scale if we're going to attempt biosphere-wide instrumentation. And that means we need professional instrumentors. Put a slightly different way, to achieve the economy of scale, biosphere instrumentation can't just be limited to a research activity - it needs to be its own industry, driven by its own professional workforce. To trigger the scientific revolution we want, we need to build an entire industry.
\newline

\section{Conclusion}

If we're going to be sophisticated stewards of our planet, we need a scientific revolution in biosphere modeling. Scientific revolutions in the past were driven by revolutions in instrumentation. Today is no exception. Our strategy? Using the 5 dimensions of human well being and the 25 ecosystem service categories outlined by the Millennium Ecosystem Assessment as a guide, find the researchers already building the instrumentation we need and create professional industries dedicated to deploying and maintaining that instrumentation at scale. Let's create a biosphere instrumentation industry - let's create Network Earth.
\newline

\textit{December 2022}

\chapter{Biological Control}
\section{The Helpers Among Us}
It was the late 1860's. While coming in at only 5mm the cottony cushion scale \textit{Icerya purchasi} was having an oversized effect on California's citrus industry \cite{ahajek}. A likely traveler from Australia, this little bug found California to be a veritable paradise, and by the 1880's had spread throughout California and was such an irritation that farmers were burning their own orchards in an attempt to control the bug. Enter the vedalia beetle \textit{Rodolia cardinalis}. Also from Australia, this predator of the cottony cushion scale was imported on purpose at the request of Charles Riley - the head of entomology for the US government at the time. The idea was that the cottony cushion scale was experiencing a release from predation pressure and therefore was so successful at decimating the California citrus industry. The solution? Bring in the predator it had at home. By the 1890's the cottony cushion scale was under control and the whole thing had cost not more than \$5000! And thus the field of biological control was born.
\newline

There are, roughly speaking, three categories of biological control. The first, and oldest, is exemplified by the vedalia beetle - \textbf{classical} biological control. In classical biological control the idea is to introduce a foreign biological control agent for the purpose of \textit{permanent} establishment and control \cite{ahajek} of a pest. Contrast this with \textbf{augmentation} where the intention is never to actually create a permanent population but instead inundate an area with finite and temporary deluges of biological control agents. Finally there is \textit{conservation} and \textbf{enhancement} wherein instead of focusing on foreign or exotic predators, the idea is to buttress the locally available biological control agents instead. 
\newline

Each of these three strategies has seen use throughout the globe. We already saw an example of classical biological control in the vedalia beetle. The tiny parasitic wasp \textit{Anagyrus lopezi} is an example of augmentation. Every year it is mass bred in the millions in order to be released to control the mealybug - a cassava pest \cite{wpark}. Such work saves an estimated \$14 billion each year. Finally farmers take advantage of a fungal pathogen of cotton aphids \textit{Aphis gossypii} to control aphid populations \cite{ahajek}. Given this fungal pathogen is a native, we find ourselves with an example of conservation and enhancement. Another obvious example of the importance of conservation biological control is that the very pesticides we use to reduce pests often also take out local natural enemies. The reduction in predation pressure then results in a whole gamut of new pests becoming an issue \cite{ridgway}. Many are the unsung heroes around us.

Given all of this it is no wonder that biological control is listed as one of the regulating ecosystem services in the Millennium Ecosystem Assessment \cite{mas}. So, in the spirit of Network Earth, let's dig into this one, understand what kinds of modeling needs to be done and where we can help.

\section{Outcomes}
First we must start, of course, with the elements of human well being. To understand what kind of modeling is required, necessitates that we understand what our outcomes are. So let's briefly review each of the five elements.

\subsection{Necessary Material}
The list of living things that we depend upon for our base needs is pretty endless. Every food you eat had to be grown. If that food is an animal, that animal had to eat too. Many of the additives we use from corn syrup to palm oil have their foundations in plants. If it contains wool or cotton, the clothing on your back comes from a plant or animal source as well. If there's wood framing your house or wood in your furniture, that had to be grown too. And if you heat with things like wood chips or firewood (which is especially prevalent in the developing world) that too came from a biological source.
\newline

For all of their variety, all biological things share one thing in common - they all are part of ecosystems. Regardless of whether you are a plant or an animal, whether you are domesticated or not, there are things that you eat, things that eat you, things that give you disease, and/or things that parasitize you. As a result for every crop we grow or animal we raise there are pests or critters that given the opportunity would become pests. Indeed in a study of Californian alfalfa, they found that in a single unsprayed field there averaged 1,000 \textit{different} species of insect alone! 
\newline

Controlling pests is therefore absolutely essential to providing the necessary material for a fulfilling life. And biological control is one mechanism for doing just that.

\subsection{Health}
Health is subject to so many things. But for our purposes two things in particular stand out - nutrition and disease. The first we've already talked about but the second is also relevant to biological control. 
\newline

Disease comes to us by virtue of vectors - disease carrying mechanisms. And vectors can be all sorts of things. They can be a friend who sneezes at your party. They can be an unclean door handle. But in many cases diseases, vectors come in the form of pests. From rats and plague to mosquitoes and the Zika virus - pest vectors abound. It should therefore come as no surprise that biological control is also important for regulating disease. Remove the natural predators of mosquitoes and all mosquito borne diseases are given freer range. Remove the predators that regulate rats and the same happens with them. Keeping pests in check not only helps provision our world but also helps control disease.

\subsection{Social Relations, Security, Freedom and Choice}
For me at least, pest control's impact on food security and abundance, disease control, and the supply of fuel and building materials is a "enough said" for the rest of these categories. If you struggle to get food, heat, or shelter, what kind of freedom do you really have?
Choice is the privilege of those who've had their basic needs met. Most of the social traditions I'm familiar with center around food and drink in one way or another and, hell, what worse way to disrupt social relations that to end up bedridden because of infectious disease? Food, shelter, and health provide the bedrock for everything else we do in life - so these as outcomes seem like a perfectly good set to me.

\section{Models}
Pest control is ultimately a question of population dynamics \cite{nmills} - if you want to control a pest population you'd better be able to model populations. Fortunately or unfortunately - depending on your point of view - when it comes to population modeling there's \textit{a lot} going on. 
\newline

Most population modeling relevant to biological control has centered around host-parasite dynamics \cite{ahajek} due to it's relative simplicity. Yet using the word simple in this context seems wildly off. To get a sense of this, all one needs to do is review the literature on modeling this simple system and look at all the different kinds of assumptions people have tried to pull in \cite{nmills}. Some models make the assumption that populations are relatively continuous (leading to differential equations). Others have acknowledged the fact that insect often come in discrete cycles (leading to difference representations). Some assume synchrony between parasites and their hosts while others acknowledge asynchronous relationships. Some models assume hosts are parasitized directly whereas others deal with egg parasitization. Speaking of eggs, some models assume infinite egg laying capabilities whereas others assume finite capacity per female. Then there's the fact that parasite efficacy is a not-always-linear function of host density. This so called functional response is itself a point of debate among scientists with at least three different types of functional response being common. Then there's the question of whether hosts are parasitized in a uniform or heterogeneous way and whether that heterogeneity is found in space, time, or both! If this is not complicated enough insects can often control sex distributions on the basis of prey density, perform very differently depending on nutrition \cite{ridgway}, compete among each other for hosts, and in some cases differ in their taste for prey within the same species! Now take all of this complexity and apply it across the hundreds of arthropods \cite{nmills} that have been introduced as part of biological control problems and you're left with one whopper of a modeling problem to solve. 
\newline

At this point you may be thinking to yourself - do we really need to model all of this? Why can't we just do what they did with the vedalia beetle and just find each pest's natural predator, introduce it, and call it a day? The answer is simple - the vedalia beetle example is an exceptionally lucky one. Of the 3600 introductions across the 500 distinct biological control species that were introduced between the 1880's and the 1990's only 30\% actually took root and of those 30\% only 36\% have resulted in meaningful control!
\newline

The issues are numerous and relate back to the modeling. Beetles introduced to mono-cropped fields had no habitat to live in and so never made it to the center of those fields \cite{ahajek}. Wasps without pollen to feed on end up starving away. Overly effective predators decimate host populations which then leads to their own demise, necessitating new introductions year after year. Predators effective in one region of the world end up positively useless in another. 
\newline

Yet, this doesn't mean biological control is useless. Adding rows of natural vegetation (coined beetle banks) into fields allowed those beetles to become effective predators. Mass production of insects can help augment otherwise insufficient natural populations. Selecting more fecund versions of a species can allow for better biological control. And so solutions to all of these problems exist, but they can only be found, optimized, and implemented correctly with the help of population models. However didn't we just say loads of modeling has been done? If there's been so much modeling work, why are the results still so poor? The answer to that question is in what the modeling work has been about. 
\newline

Biological control contains a bit of a paradox - in order to actually control a pest population you need a less than perfect predator. The reason for this is pretty clear - if a predator removes enough of the host so that it can no longer maintain its own population, then once the predator has killed itself off the pest is free to increase back to epidemic levels. Most early population models kept finding this - introduced predators should lead very quickly to no predators at all. Yet the vedalia beetle demonstrated that this was not the case - stability could be found. And so much of the population modeling to date has centered around understanding this question - how does stability arise. But much of this modeling has been conceptual - its purpose has been to give an explanation for population stability, not to provide very specific predictions for any particular species. As a result of this focus on conceptual modeling, much of the actual application of biological control has been more... bespoke... in nature. It follows then that if we want to improve the efficacy of biological control we need to develop \textit{predictive} models of population dynamics. And this is where Network Earth fits in. Predictive models require lots data and that is exactly what Network Earth is all about.

\section{Data}
To understand what kind of data we're going to need, we first need to understand what kind of model we will be using. We already know that there are a lot of factors that could come into play here - egg laying limitations, nutrition, host finding efficacy, host parasite synchrony, etc. - in addition to the obvious host and parasite densities. But, if we think about it, all of these components are effectively just rules to a game. Parasites, hosts, host plants, and all the others are just players on a board following specific rules and limitations. And our interest is in which of these rules, players, and limitations are responsible for their population dynamics. As a result, simulation seems the model of choice here as simulations are nothing more than a specification of objections, their playbooks, and the resulting interactions. 
\newline 

The first question is who are the players and how are they represented. This is where the complexity and transferability of our simulation gets largely determined. For example if we tried to simulate the physical universe using electrons, protons, and neutrons we'd have a much more complicated simulation than we actually need. Even just jumping up to atoms provides an order of magnitude of simplification. So in order to keep the complexity at reasonable levels we want to model the most circumscribing objects we can without sacrificing precision. Generally, higher order groupings (like atoms instead of elementary particles) can be found by modeling the underlying groupings in more limited scenarios and then noting the emergent properties that result. 
\newline

Transferability is also an important consideration here. If we build simulations species by species there's going to be a lot of duplication of effort. For example, different species of wasp share common characteristics, it's just that the specifics change. Things like host preference, size, host finding efficacy, and the like can be parametrized so that one general rule set covers a whole series of different taxa. Then, in building one simulation we effectively simulate them all. All that's required is then filling in their specific set of parameters.
\newline

The second question is what are the rules. Similar to what we just said regarding the players, parametrization is important here in order to drive transferability. But equally important is minimization. We want to find the smallest set of rules required to get the results we are looking for. Why? Because it makes it far easier for us to interpret the simulation. Remember that at the end of the day we want to use these simulations to optimize applications of biocontrol and in order to do that we need to understand why things do the things they do. The more complicated a system becomes the harder that is. So, we want to pare things back as close to the basics as we can.
\newline

Finally we need the data in order to test all of this. That means gathering loads of population dynamics data. Without this kind of data any attempt at modeling is just spitting in the dark. However there's an immediate difficulty here - populations change in real time, which means gathering useful data can take months or years. Thankfully we have a couple of tricks up our sleeves. First of all, by parametrizing as we've done, we can collect tons of examples by just stratifying across different species. And then, we can exchange space for time. Individual population "experiments" might take months to years, but the natural world is carrying out gazillions of them all the time. If we can capture as much breadth and depth in space as possible we'll be able to capture loads of population "experiments" in the same amount of time as it would take to capture one. This is how we can achieve scale and variability.
\newline

Alright so let's pull this all together. First, we're going to want to build simulations. To understand what kinds of objects and rules are candidates for simulation, we're going to need to gather life history and behavioral data on our parasites, hosts, and any competitors or intermediaries. These may or may not get directly used in the simulation, but even if we do find emergent objects and properties it will come from studying this lower level data. Second, we're going to want to set up our simulation in a parametrized way so that we can describe as wide array of host-parasite interactions as possible with the same model. This will necessitate gathering the aforementioned host and parasite data in a generalized and standard fashion. Finally we're going to need loads of population data in order to test our various attempts at simulation. In order to get this kind of data we'll need to implement broad, dense instrumentation to trade space for time and make sure the instrumentation covers as wide a range of taxa as possible. 
\newline

So, in the spirit of network earth and not rebuilding the wheel, our first step then is to understand what kinds of instrumentation already exists and how we can help bring it up to scale. Time to get cracking.
\newline

\textit{January 2023}

\bibliographystyle{plain}
\bibliography{reference}
\end{document}